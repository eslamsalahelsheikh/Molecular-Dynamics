\chapter{Implementation}\label{chap:Implementation}

\section{Velocity-Verlet integrator}
\subsection{Test strategy for Verlet integrator}
We test the Verlet integrator by comparing the results of the Verlet integrator with the results of the analytical solution of the equations of motion of a particle. the analytical solution of the equations of motion of a particle is given by the following equations:
\begin{equation}
\label{eq:analytical}
    \begin{aligned}
    x_i(t+dt) &= x_i(t) + v_i(t)dt + \frac{1}{2m}f_i(t)dt^2 \\
    v_i(t+dt) &= v_i(t) + \frac{1}{2m_i}(f_i(t)+f_i(t+dt))dt
    \end{aligned}
\end{equation}
If we assume that the there is no acting forces (constant and equal to zero) on the particles, then the analytical solution of the equations of motion of a particle after N time steps is given by the following equations:
\begin{equation}
\label{eq:analytical2}
    \begin{aligned}
    x_i(t+N*dt) &= x_i(t) + \sum_{i=0}^{N}v_i(t)dt \\
    v_i(t+N*dt) &= v_i(t)
    \end{aligned}
\end{equation}
where $dt$ is the time step, $N$ is the number of time steps, $x_i(t)$ is the position of the particle $i$ at time $t$, $v_i(t)$ is the velocity of the particle $i$ at time $t$, $f_i(t)$ is the force acting on the particle $i$ at time $t$, and $m_i$ is the mass of the particle $i$.

That means if we compare the result after all the integration steps of the two Verlet steps with the expected output of the analytical solution, then we can be sure that the Verlet integrator is working correctly.

\section{Lennard-Jones potential force}
\subsection{Derivation of the analytical expression for the forces of the Lennard-Jones potential}

\section{Berendsen thermostat}
\subsection{Test strategy for Berendsen thermostat}

\section{Embedded atom method}
\section{units and specification of the time unit}
\subsection{time step for the gold potential}


\section{Neighbor List}

\section{Parallelization using MPI}

Explain the math and introduce notation.
\begin{algorithm}[p]
\caption{Stochastic Gradient Descent: Neural Network}
\label{alg:backpropnn}
\begin{algorithmic}
    % \ttfamily
    \State Create a mini batch of $m$ samples $\vec{x}_0 \ldots \vec{x}_{m-1}$
    \ForEach{sample $\vec{x}$}
        \State $\vec{a}^{\vec{x},0} \gets \vec{x}$  \alignedComment{Set input activation}
        \ForEach{Layer $l \in \{1\ldots L-1\}$}  \alignedComment{Forward pass }
            \State $\vec{z}^{\vec{x},l} \gets \mathbf{W}^l \vec{a}^{\vec{x},l-1}+\vec{b}^l$
            \State $\vec{a}^{\vec{x},l} \gets \varphi(\vec{z}^{\vec{x},l})$
        \EndFor
        \State $\bm{\delta}^{\vec{x},L} \gets \nabla_{\vec{a}} C_\vec{x} \odot \varphi'(\vec{z}^{\vec{x},L})$ \alignedComment{Compute error}
        \ForEach{Layer $l \in L-1, L-2 \ldots 2$}  \alignedComment{Backpropagate error}
            \State $\bm{\delta}^{\vec{x},l} \gets ((\mathbf{W}^{l+1})^T \bm{\delta}^{\vec{x},l+1})\odot \varphi'(\vec{z}^{\vec{x},l})$
        \EndFor
    \EndFor
    \ForEach{$l \in L, L-1 \ldots 2$} \Comment  \alignedComment{Gradient descent}
        \State $ \mathbf{W}^l \gets \mathbf{W}^l-\frac{\eta}{m} \sum_\vec{x} \bm{\delta}^{\vec{x},l} (\vec{a}^{\vec{x},l-1})^T$
        \State $\vec{b}^l \gets \vec{b}^l-\frac{\eta}{m}\sum_\vec{x} \bm{\delta}^{\vec{x},l}$  
    \EndFor
\end{algorithmic}
\end{algorithm}


\begin{figure}[t]
    \begin{center}
    \begin{tikzpicture}
        \node (a) at (0,0) {a};
        \node (b) at (2, 0) {b};
        \draw[->] (a) -- (b);
        

    \end{tikzpicture}
    \end{center}
    \caption[Tikz Example]{Use tikz to draw nice graphs!}
    \label{fig:Tikz}
\end{figure}