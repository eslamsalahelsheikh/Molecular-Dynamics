\chapter{Implementation}\label{chap:Implementation}

\section{Velocity-Verlet integrator}
\subsection{Test strategy for Verlet integrator}
We test the Verlet integrator by comparing the results of the Verlet integrator with the results of the analytical solution of the equations of motion of a particle. We use the following equations of motion for a particle:
\begin{equation}
    \frac{d^2 x}{dt^2} = -\frac{k}{m}x
\end{equation}
\begin{equation}
    \frac{d^2 v}{dt^2} = -\frac{k}{m}v
\end{equation}
where $x$ is the position of the particle, $v$ is the velocity of the particle, $m$ is the mass of the particle, and $k$ is the spring constant. We use the following initial conditions:
\begin{equation}
    x(0) = 1
\end{equation}
\begin{equation}
    v(0) = 0
\end{equation}
\begin{equation}
    x'(0) = 0
\end{equation}
\begin{equation}
    v'(0) = 1
\end{equation}
where $x'$ is the velocity of the particle, and $v'$ is the acceleration of the particle. We use the following values for the mass of the particle and the spring constant: 
\begin{equation}
    m = 1
\end{equation}
\begin{equation}
    k = 1
\end{equation}
We use the following values for the time step:
\begin{equation}
    \Delta t = 0.01
\end{equation}
We use the following values for the number of steps:
\begin{equation}
    N = 1000
\end{equation}
We use the following values for the initial time:
\begin{equation}
    t_0 = 0
\end{equation}
We use the following values for the final time:
\begin{equation}
    t_f = 10
\end{equation}
We use the following values for the initial position:
\begin{equation}
    x_0 = 1
\end{equation}
We use the following values for the initial velocity:
\begin{equation}
    v_0 = 0
\end{equation}
We use the following values for the initial acceleration:

\section{Lennard-Jones potential force}
\subsection{Derivation of the analytical expression for the forces of the Lennard-Jones potential}

\section{Berendsen thermostat}
\subsection{Test strategy for Berendsen thermostat}

\section{Embedded atom method}
\section{units and specification of the time unit}
\subsection{time step for the gold potential}


\section{Neighbor List}

\section{Parallelization using MPI}

Explain the math and introduce notation.
\begin{algorithm}[p]
\caption{Stochastic Gradient Descent: Neural Network}
\label{alg:backpropnn}
\begin{algorithmic}
    % \ttfamily
    \State Create a mini batch of $m$ samples $\vec{x}_0 \ldots \vec{x}_{m-1}$
    \ForEach{sample $\vec{x}$}
        \State $\vec{a}^{\vec{x},0} \gets \vec{x}$  \alignedComment{Set input activation}
        \ForEach{Layer $l \in \{1\ldots L-1\}$}  \alignedComment{Forward pass }
            \State $\vec{z}^{\vec{x},l} \gets \mathbf{W}^l \vec{a}^{\vec{x},l-1}+\vec{b}^l$
            \State $\vec{a}^{\vec{x},l} \gets \varphi(\vec{z}^{\vec{x},l})$
        \EndFor
        \State $\bm{\delta}^{\vec{x},L} \gets \nabla_{\vec{a}} C_\vec{x} \odot \varphi'(\vec{z}^{\vec{x},L})$ \alignedComment{Compute error}
        \ForEach{Layer $l \in L-1, L-2 \ldots 2$}  \alignedComment{Backpropagate error}
            \State $\bm{\delta}^{\vec{x},l} \gets ((\mathbf{W}^{l+1})^T \bm{\delta}^{\vec{x},l+1})\odot \varphi'(\vec{z}^{\vec{x},l})$
        \EndFor
    \EndFor
    \ForEach{$l \in L, L-1 \ldots 2$} \Comment  \alignedComment{Gradient descent}
        \State $ \mathbf{W}^l \gets \mathbf{W}^l-\frac{\eta}{m} \sum_\vec{x} \bm{\delta}^{\vec{x},l} (\vec{a}^{\vec{x},l-1})^T$
        \State $\vec{b}^l \gets \vec{b}^l-\frac{\eta}{m}\sum_\vec{x} \bm{\delta}^{\vec{x},l}$  
    \EndFor
\end{algorithmic}
\end{algorithm}


\begin{figure}[t]
    \begin{center}
    \begin{tikzpicture}
        \node (a) at (0,0) {a};
        \node (b) at (2, 0) {b};
        \draw[->] (a) -- (b);
        

    \end{tikzpicture}
    \end{center}
    \caption[Tikz Example]{Use tikz to draw nice graphs!}
    \label{fig:Tikz}
\end{figure}